\chapter{Social and legal relationships}\label{cha:soci-legal-relat}

One of the key aims of this project is to examine the structures of power and obligation that are created or presupposed by the institution of private property.
In this chapter, I will set out some of the analytical tools that I use to do this.
Specifically, I will provide some vocabulary for discussing two types of phenomena:
\begin{enumerate}
  \item Relationships of obligation and accountability.
  \item The power to transform such relationships. 
\end{enumerate}
These two sorts of phenomena play key roles in shaping the normative terrain of a society.

I borrow some of these tools from W. N. Hohfeld's, seminal essay, \emph{Some
fundamental legal relations in judicial reasoning}
\citeyearpar{hohfeldfundamentallegal1913}.
In this essay Hohfeld sets out several basic legal relationships and defines
them in terms of one another \citep[pp.
28--59]{hohfeldfundamentallegal1913}

Hohfeld's analytical tools are best suited to examining legal systems, and moral systems and social arrangements that mirror the normative structure of legal systems. 

As I said, in Chapter 1, there are many ways to model and conceive of moral and
social relationships.
% * TODO write an appropriate section and include a direct reference. %

\section{Duties and claims}\label{sec:duties-claims}
A central feature of moral life in communities is the notion that there are
certain things that we owe to one another, and certain things that other people can claim from us as their due.
This generally, is what we speak about when we talk about duties and claims.
Philosophers often argue about what precisely what features are necessary for duties
% * Duties to others
% *

\subsection{Relationships of obligation and accountability}\label{sec:obl-acc}
There are many ways to model these relationships.
Philosophers


\section{Legal powers}\label{sec:legal-powers}
\begin{quote}
  A change in a given legal relation may result (1) from some superadded fact or
  group of facts not under the volitional control of a human being (or human
  beings); or (2) from some superadded fact or group of facts which are under the
  volitional control of one or more human beings.As regards the second class of
  cases, the person (or persons) whose volitional control is paramount may be said
  to have the (legal) power to effect the particular change of legal relations
  that is involved in the problem.
\citep[44]{hohfeldfundamentallegal1913}


\end{quote}

\section{Formal tools}\label{sec:formal-tools}

\subsection{Hohfeld's taxonomy of basic legal relationships}\label{sec:hohf-taxon}
In his seminal two part essay, , W.
N. Hohfeld \citep{hohfeldfundamentallegal1913,hohfeldfundamentallegal1917} sets
out to provide precise definitions of several basic legal concepts, as an
antidote for what he sees and sloppy legal reasoning.
Most famously, he defines two sets of four types of legal relationships in terms
of their logical relationships with one another.
These definitions provide a helpful way to examine the formal structure of legal
systems and other moral and social arrangements.

Hohfeld's first tetrad of legal relationships can be used to model the legal duties that persons owe one another under specific legal systems.
It is organized around the idea of a legal duty that one person has to another person to perform or omit some action.
According to Hohfeld's taxonomy, this duty corresponds to a claim that the latter bears against the former.
Specifically, the claim that the former perform or omit the relevant action.
\footnote{Hohfeld called these legal claims, \emph{rights}, and he used the term \emph{right} exclusively for this sort of legal relationship. Common usage supports using the term \emph{right} much more broadly, so I prefer to use the term or \emph{claim right} or simply \emph{claim} to refer to this more specific type of legal relationship.\label{fn:rightclaim}}
So if, for instance, Sarandon has a legal duty to Davis not to drive Davis off of a cliff, then Davis has a claim against Sarandon, that Sarandon not drive her off of a cliff. 
However, if Sarandon has no legal duty to Davis not to drive Davis off of a cliff,\footnotemark{} then Davis has no legal claim against Sarandon that Sarandon not drive her off of a cliff.
\footnotetext{Hohfeld called this sort of lack of a duty a \emph{privilege}, and many of those who have adopted the basic taxonomy of legal relationships have called it a \emph{liberty}.
  However, both of these terms strike me as problematic for reasons I shall discuss below.}


% \setlength{\belowdisplayskip}{0pt} \setlength{\belowdisplayshortskip}{0pt}
% \setlength{\abovedisplayskip}{0pt} \setlength{\abovedisplayshortskip}{0pt}
More formally, given any two persons, \emph{A} and \emph{B}, and any action or omission, $\phi$:
  \begin{align}
  \text{\emph{A} has a duty to \emph{B} to }\phi\: \veebar\: \text{\emph{A} has no duty to \emph{B} to }\phi.\\
  \text{\emph{B} has a claim on \emph{A} that \emph{A} }\phi{}\: \veebar\: \text{\emph{B} has no claim on \emph{A} that \emph{A} }\phi{}.\\
  \text{\emph{A} has a duty to \emph{B} to }\phi \iff \text{\emph{B} has a claim on \emph{A} that \emph{A} }\phi.\\
   \text{\emph{A} has no duty to \emph{B} to }\phi \iff \text{ \emph{B} has no claim on \emph{A} that \emph{A} }\phi.
   \end{align}

This consists of these four legal relationships:
\begin{description}
  \item[Duty] A duty to another person to perform (or omit) a particular action.
  \item[Claim] A claim against another person that they perform (or omit) a particular action.
  \item[No duty] No duty to another person to perform (or omit) a particular action.
  \item[No claim] No claim against another person that they perform or omit a particular action. 
\end{description}
   
Hohfeld and many of those who have adopted his basic taxonomy of legal relationships, have given these legal relationships special names.
These names are often confusing, because, in ordinary legal parlance, they carry implications that do not hold under Hohfeld's, rather minimal, definitions of these terms. 

(A has a duty to B to $\phi$). (B has a claim against A that A $\phi$)

% * TODO Add section on duties
% * TODO Make shortcut for filling paragraphs by sentence.

% Local Variables:
% mode: latex 
% TeX-master: "../An_analysis_of_private_property"
% End:

