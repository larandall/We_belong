\chapter{Basic philosophical commitments}
\label{chap:basic-comm}
I want to start this essay by setting out some basic features of my approach to
moral philosophy.
One of the most important features of my approach is a form of pragmatic pluralism
about both ontology and ethics.
The gist of this position is that the world itself does not fully determine either
a singular true portrayal of the world, or a single correct set of ethical
beliefs.
Human beings construct their understandings of the world and their ethical views
through active engagement with the world.
This process is shaped both by deep features of the world itself and by the
culture, practices, aims, and history of the community and individuals that are
seeking to understand the world or to make sound ethical judgments.
Consequently, I think that different communities might come to understand the
world, in very different ways, and nonetheless have views that reflect deep
features of the world and serve the purposes for which they are seeking to
understand the world.
Likewise, I think that different communities or individuals might come to adopt
very different sets of ethical beliefs and practices which all reflect deep
ethical truths and all constitute sound ethical practice.
This implies that there may be more than one correct, (or at least reasonable)
understanding of the world, and that their may be more than one sound set of
ethical beliefs and practices.
It does not imply, however, that all beliefs about the world or all sets of
ethical belief and practice are equally valid or reasonable.
Some models of the world are more accurate or more useful than others, and some
sets of ethical belief and practice are more consistent with deep ethical truths
than others.

I will explain these positions in more detail in the following sections.
First I will explain the particular flavor of ontological pluralism that informs
my ethics and metaethics in \refse{sec:ont_plural}.
After this I will explain how I understand moral reasoning, in
\refse{sec:nat_mor}. This will enable me to defend a form of ethical and
metaethical pluralism in \refse{sec:eth_plur}.
\section{Ontological pluralism}
\label{sec:ont_plural}
There is no uniquely correct way to understand the world.
When we seek to understand the world, we use concepts that we have inherited from
others or we develop our own.
These concepts give us ways to understand and interact with the world
effectively.
We use these concepts to model the world and to chop it into manageable bits that
we can understand.
So, for instance, when I look at the corner my desk, I do not simply see a series
of meaningless shapes and colors; I see a colorful lamp and a pine cone that I
put there to make my space feel warm and inviting.
\emph{Lamp}, \emph{pine cone}, and \emph{desk} are useful concepts for me because they provide a
meaningful way for me to model and interact with my world.
It is useful to think of lamps, pine cones, and desks both as wholes and as
distinct and persistent physical objects, because I interact with them as
distinct and persistent physical objects.
These concepts also constitute meaningful categories because they reflect
important, recognizable patterns in my world (many trees in the pacific northwest
produce seed-bearing cones), or they reflect ways in which people regularly
interact with objects in my world (people make lamps and desks and       use them to
light their rooms or to write on).
These categories reflect both features of the objects themselves and ways that
It is useful to think of them as different types of physical objects because they
I can arrange them in my room and  elp me to make sense of my experience.
They are all generic concepts that allow me to categorize a variety of distinct,
but similar objects.
These categories help me to understand my world and interact with it effectively.
\emph{Lamp}, \emph{pine cone}, and \emph{desk} are useful concepts because they are all
\section{On the nature of morality and moral reasoning}
\label{sec:nat_mor}
\section{Ethical and metaethical pluralism}
\label{sec:eth_plur}
There are at least four basic philosophical commitments that inform my approach.

I want to answer the question, what are we doing when we are doing moral
philosophy.
\chapter{Foundational ethical commitments}
\label{ch:eth_com}
